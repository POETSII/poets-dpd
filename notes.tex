\documentclass{article}

\usepackage{amsmath}
\usepackage{amssymb}
\usepackage{algorithmicx}
\usepackage{algpseudocode}


\newcommand{\leftarrowplus}{\xleftarrow{+}}
\newcommand{\dt}{\Delta t}

\newcommand{\Time}{\mathbb{R}}
\newcommand{\Bead}{\mathbb{N}}
\newcommand{\Pos}{\mathbb{R}^3}
\newcommand{\Vel}{\mathbb{R}^3}
\newcommand{\Force}{\mathbb{R}^3}

\newcommand{\VEC}[1]{\vec{\mathbf #1}}
\newcommand{\IND}[3]{\vec{\mathbf #1}\left[#2,#3\right]}


\begin{document}


Fundamental parameters:
\begin{itemize}
\item $l$ : Length of box in each dimension
\item $n$ : Total number of beads
\end{itemize}

Derived parameters:
\begin{itemize}
\item $d = l^3 / n$ : Average beads per unit volume (density)
\end{itemize}

World description:
\begin{itemize}
\item $i=1..n$ : Bead index
\item $species(i) -> a$ : Gives the species (bead type) of bond i 
\item $interaction(a,b) -> (c,d)$ : Gives the conservative (c) and dissipative force (d) between species a and b.
\item $bond(i,j) -> (kappa,r0)$ : Gives the bond strength (kappa) and preferred distance (r0) between i and j
\item $angle(i,j,k) -> (kappa,phi0)$ : Gives the angle strength (kappa) and preferred angle (theta0) between i,j,k`
\end{itemize}

The description uses a sparse map to describe bonds and angle pairs. If there
is no bond or connection, then kappa is 0. We define the helpers:
\begin{itemize}
\item $is_bonded(i,j) == bond(i,j)_.{kappa} > 0$
\item $is_angled(i,j,k) == angle(i,j,k)_{kappa} > 0$
\end{itemize}

Requirements are:
\begin{itemize}
\item $bond(i,j) == bond(j,i)$ : Bonds must be symmetric
\item $angle(i,j,k) == angle(k,j,i)$ : Angles are symmetric around centre
\item $is_angled(i,j,k) \rightarrow is_bonded(i,j) \hat  is_bonded(j,k)$ : angles must appear over bond pairs:
\item $is_bonded(i,i) = false$ : Bonds cannot be reflexive (self-bonded)
\end{itemize}

Some useful static properties are:
\begin{itemize}
\item $bonds(i) = { (i,j) : 1 \leq i \leq n \hat is_bonded(i,j) \hat i < j }$ : Set of bonds for bead i.
\item $\overline{bonds} =  \frac{1}{n} \sum_{i=1}^n |bonds(i)|$ : Average bonds per bead. 
\item $bonds^+ =  \max_{i=1}^n |bonds(i)|$ : Maximum bonds per bead. 
\item $angles(i) = { (i,k) : 1 \leq i < k \leq n \hat is_angle(i,j,j) }$ : Set of angle partners for bead j.
\item $\overline{angles} = {1}{n} \sum{j=1}^n |angles(j)|$ : Average angles per bead.
\item $angles^+ = \max_{j=1}^n |angles(j)|$ : Maximum angles per bead.
\end{itemize}


We model the simulation state using three mappings which map bead indices
and times to values. These can be viewed as arrays, or as sparsely defined
functions.
\begin{itemize}
\item $\IND{x}{i}{t}$ : Position of bead i at time t.
\item $\IND{v}{i}{t}$
\item $\IND{f}{i}{t}$
\end{itemize}
All state mappings must be written to before they are read, and
$x$ and $v$ must be written to exactly once.
We treat $f$ as an accumulator, so $\IND{f}{i}{t}$ starts as
0, and we can accumulate to it up until it is read in a 
non-accumulation context. We use the shorthand:
$ \IND{f}{i}{t} \leftarrowplus \VEC{v} <=> \IND{f}{i}{t} \leftarrow \IND{f}{i}{t} + \VEC{v}$.

Some useful dynamic properties are:
\begin{itemize}
\item $\operatorname{nhood}(i,t) = { j : 1<=j<=n \hat i \neq j \hat ||x(i)-x(j)||_2 < 1}$
\end{itemize}

The overall simulation algorithm is then:
\begin{algorithmic}
\While{$t < T$}

    \For{i in 1..n}
          \State{$\IND{x}{i}{t+\dt} \leftarrow \IND{x}{i}{t} + \IND{v}{i}{t} \dt  +  \IND{f}{i}{t} \dt^2/2$}
        \State{$\IND{v}{i}{t+\dt/2} \leftarrow \IND{v}{i}{t} +  \IND{f}{i}{t} \dt/2$}
    \EndFor

    \For{i in 1..n}
        \For{j in nhood(i,t)}
            \State{ $\IND{f}{i}{t+\dt} \leftarrowplus \operatorname{pair\_force}\left(t,\ i, \IND{x}{i}{t+\dt}, \IND{v}{i}{t+\dt/2},\ j, \IND{x}{j}{t+\dt}, \IND{v}{j}{t+\dt/2} \right)$}
        \EndFor
        \For{$(a,c) in angles(i)$}
            \State{$(f(a,t+\dt),f(i,t+\dt),f(c,t+\dt)) \leftarrowplus \operatorname{angle\_force}(a, x(a,t), i, x(i,t), c, x(c,t) )$}
        \EndFor
    \EndFor

    \For{$i in 1..n$}
        \State{$\IND{v}{i}{t+\dt} \leftarrow \IND{v}{i}{t+\dt/2} +  \IND{f}{i}{t+\dt} \dt/2$}
    \EndFor

    \State{$t \leftarrow t + \dt$}
\EndWhile
\end{algorithmic}

\begin{algorithmic}
\Function{pair\_force}{$t:\Time, i:\Bead, x_i:\Pos, v_i:\Vel, i:\Bead, x_i:\Pos, v_i:\Vel$}
	\State{${\bf r} \leftarrow x_i - x_j$}
	\State{$r \leftarrow || {\bf r} ||_2 $}
	\If{$r \geq 1 \vee  r \leq \epsilon$}
		\State{\textbf{return} 0}
	\EndIf
  \State{$(c,d) \leftarrow interaction(species(i),species(j))$}
	\State{...}
	 \State{\textbf{return} ${\bf r} / r ( f_c + f_d + f_r + f_b ) $}
\EndFunction
\end{algorithmic}


\section{Handling angles}

Bonds can be handled in a natural way in an event-driven DPD simulation,
as all the information needed to calculate a bond's contribution to the
pair-wise force is available while calculating the DPD pair-wise forces.
The bond's contribution to the force is relatively cheap to calculate,
effectively adding just one $\kappa (r-r_0)^2$ term to the forces,
requiring just three scalar operations. As a consequence it is often cheap enough
to just do the bond force calculation for all pair-wise interactions, and
keep $\kappa$ at zero for non-bonded pairs. This does requires that
we have a cheap method for evaluating $bond(i,j)$ at run-time, or
at least a method that is cheap for the vast majority of cases
where $is_bonded(i,j)$ is false.

The angle bonds present much more of a challenge in an event-driven
DPD simulation, because:
\begin{enumerate}
\item The head and tail of the angle will always receive the position of the middle
  bead, but will often not receive the position of the far partner.

\item the information about partner beads in neighbouring cells will not
  arrive at the centre bead's cell at the same time, or in a predictable order.
\end{enumerate}



In an event-driven approach we receive beads from neighbouring cells
in a non-deterministic order, but in order to 

\end{document}
